\subsection{Lubrication Approximation}
As shown in \cite{kondic2003instabilities}, if the Reynold's number is low, the Navier-Stokes equations under the effect of gravity can be reduced to 
\begin{gather}
    \grad_2 p = \mu \frac{\partial^2 \vect{v}}{\partial z^2} + \rho g \sin\alpha \vect{i} 
    \label{eq:pressure_grad}\\
    \pderiv{p}{z} = -\rho g \cos \alpha
    \label{eq:pressure_z}
\end{gather}
where $\grad_2 = \lrp{\partial_x, \partial_y}$ and $\vect{v} = \lrp{u, v}$.

The Laplace-Young boundary condition states that at the interface $z = \func{\phi}{x, y}$, the pressure is given by 
$\func{p}{\phi} = -\gamma \kappa + p_0$, where $\kappa$ is the curvature of the boundary, $\gamma$ is the surface tension, and $p_0$ is the atmospheric pressure. 
Thus, integrating \cref{eq:pressure_z} gives 
\begin{align}
    \nonumber \int_{\phi}^{z} \pderiv{p}{z} \; dz &=  \int_{\phi}^{z} -\rho g \cos \alpha \; dz\\
    \nonumber \func{p}{z} - \func{p}{\phi} &= -\rho g \cos \alpha \lrp{z - \phi}\\
    \func{p}{z} &= -\rho g \cos \alpha \lrp{z - \phi} -\gamma \kappa + p_0.  
    \label{eq:laplace_young}
\end{align}
When considering further conditions along the boundaries, it is helpful to break \cref{eq:pressure_grad} into its components
\begin{gather}
    \pderiv{p}{x} = \mu \pderivtwo{u}{z} + \rho g \sin\alpha 
    \label{eq:pressure_x}\\
    \pderiv{p}{y} = \mu \pderivtwo{v}{z}
    \label{eq:pressure_y}
\end{gather}
and then apply the BCs
\begin{gather}
    \evalat{\vect{v}}{z = s\lrp{x,y}} = \vect{0}
    \label{eq:bound_noslip}\\
    \evalat[\Big]{\pderiv{u}{z}}{z = \phi\lrp{x,y}} = \evalat{\pderiv{v}{z}}{z = \phi\lrp{x,y}} = 0
    \label{eq:bound_stress}
\end{gather}
where \cref{eq:bound_noslip} is a no-slip BC along the surface $z = \func{s}{x,y}$ and \cref{eq:bound_stress}
is a continuous stress BC along the fluid-air boundary $z = \func{\phi}{x,y}$. 

Integrating \cref{eq:pressure_x} twice with respect to $z$ and utilizing the BCs \cref{eq:bound_noslip} and \cref{eq:bound_stress} gives
\begin{align}
    \nonumber \int_{s}^{z}\int_{z}^{\phi} \pderivtwo{u}{z} \; dzdz &= \frac{1}{\mu}\int_{s}^{z}\int_{z}^{\phi} \lrp{\pderiv{p}{x} - \rho g \sin\alpha} \; dzdz \\
    \nonumber \int_{s}^{z} \lrp{\evalat[\Big]{\pderiv{u}{z}}{z = \phi} - \pderiv{u}{z}} \; dz &= \frac{1}{\mu} \lrp{\pderiv{p}{x} - \rho g \sin\alpha} \int_{s}^{z}\lrp{\phi - z} \; dz\\
    \nonumber \int_{s}^{z} \pderiv{u}{z} \; dz &= \frac{1}{\mu} \lrp{\pderiv{p}{x} - \rho g \sin\alpha} \int_{s}^{z} \lrp{z - \phi} \; dz\\
    \nonumber u - \evalat{u}{z = s} &= \frac{1}{\mu} \lrp{\pderiv{p}{x} - \rho g \sin\alpha}  \lrp{\frac{z^2}{z} - \phi z} \Bigg|_{s}^z\\
    u &= \frac{1}{\mu} \lrp{\pderiv{p}{x} - \rho g \sin\alpha} \lrp{\frac{z^2}{2} - \phi z - \frac{s^2}{2} + \phi s}. 
    \label{eq:vel_u}
\end{align}
Similarly, integrating \cref{eq:pressure_y} twice with respect to $z$ and using \cref{eq:bound_noslip} and \cref{eq:bound_stress} gives
\begin{align}
    \nonumber \int_{s}^{z}\int_{z}^{\phi} \pderivtwo{v}{z} \; dzdz &= \frac{1}{\mu}\int_{s}^{z}\int_{z}^{\phi} \lrp{\pderiv{p}{y}} \; dzdz \\
    \nonumber \int_{s}^{z} \lrp{\evalat[\Big]{\pderiv{v}{z}}{z = \phi} - \pderiv{u}{z}} \; dz &= \frac{1}{\mu} \lrp{\pderiv{p}{y}} \int_{s}^{z}\lrp{\phi - z} \; dz\\
    \nonumber \int_{s}^{z} \pderiv{v}{z} \; dz &= \frac{1}{\mu} \lrp{\pderiv{p}{y}} \int_{s}^{z} \lrp{z - \phi} \; dz\\
    \nonumber v - \evalat{v}{z = s} &= \frac{1}{\mu} \lrp{\pderiv{p}{y}}  \lrp{\frac{z^2}{z} - \phi z} \Bigg|_{s}^z\\
    v &= \frac{1}{\mu} \lrp{\pderiv{p}{y}} \lrp{\frac{z^2}{2} - \phi z - \frac{s^2}{2} + \phi s}. 
    \label{eq:vel_v}
\end{align}
Averaging over the height removes the $z$ dependence of $\vect{v} = \lrp{u, v}$ and gives the equation 
\begin{equation*}
    \bar{\vect{v}} = \frac{1}{h} \int_{s}^{\phi} \vect{v} \; dz
\end{equation*}
which decomposes into 
\begin{align}
    \bar{u} =  \frac{1}{h} \int_{s}^{\phi} u \; dz, \quad \bar{v} =  \frac{1}{h} \int_{s}^{\phi} v \; dz. 
    \label{eq:uv_bar_ints}
\end{align}
Solving the integrals in \cref{eq:uv_bar_ints} gives 
\begin{align}
    \nonumber \bar{u} &= \frac{1}{h} \int_{s}^{\phi} \frac{1}{\mu} \lrp{\pderiv{p}{x} - \rho g \sin\alpha} \lrp{\frac{z^2}{2} - \phi z - \frac{s^2}{2} + \phi s}  \; dz\\
    \nonumber &= \frac{1}{\mu h} \lrp{\pderiv{p}{x} - \rho g \sin\alpha} \lrp{\frac{z^3}{6} - \frac{\phi z^2}{2} - z\lrp{\frac{s^2}{2} - \phi s}} \Bigg|_{s}^{\phi}\\
    \nonumber &= \frac{1}{\mu h} \lrp{\pderiv{p}{x} - \rho g \sin\alpha} \lrp{-\frac{\phi^3}{3} + \phi^2s - \phi s^2 + \frac{s^3}{3}}\\
    \nonumber &=  \frac{1}{\mu h} \lrp{\pderiv{p}{x} - \rho g \sin\alpha} \lrp{-\frac{(h+s)^3}{3} + (h+s)^2s - (h+s) s^2 + \frac{s^3}{3}}\\
    &= -\frac{h^2}{3\mu} \lrp{\pderiv{p}{x} - \rho g \sin\alpha}
    \label{eq:u_bar}
\end{align}
and 
\begin{align}
    \nonumber \bar{v} &= \frac{1}{h} \int_{s}^{\phi} \frac{1}{\mu} \lrp{\pderiv{p}{y}} \lrp{\frac{z^2}{2} - \phi z - \frac{s^2}{2} + \phi s}  \; dz\\
    \nonumber &= \frac{1}{\mu h} \lrp{\pderiv{p}{y}} \lrp{\frac{z^3}{6} - \frac{\phi z^2}{2} - z\lrp{\frac{s^2}{2} - \phi s}} \Bigg|_{s}^{\phi}\\
    \nonumber &= \frac{1}{\mu h} \lrp{\pderiv{p}{y}} \lrp{-\frac{\phi^3}{3} + \phi^2s - \phi s^2 + \frac{s^3}{3}}\\
    \nonumber &=  \frac{1}{\mu h} \lrp{\pderiv{p}{y}} \lrp{-\frac{(h+s)^3}{3} + (h+s)^2s - (h+s) s^2 + \frac{s^3}{3}}\\
    &= -\frac{h^2}{3\mu} \lrp{\pderiv{p}{y}}. 
    \label{eq:v_bar}
\end{align}
Hence the depth averaged velocity can be represented as 
\begin{align}
    \bar{\vect{v}} = \lrp{\bar{u}, \bar{v}} = -\frac{h^2}{3\mu} \lrp{\pderiv{p}{x} - \rho g \sin \alpha, \pderiv{p}{y}} = -\frac{h^2}{3\mu} \lrp{\grad_2 p - \rho g \sin \alpha \vect{i}}.
    \label{eq:uv_bar}
\end{align}
From the Laplace-Young BC \cref{eq:laplace_young}, we find that 
\begin{align}
    \grad_2 p = \grad_2 \lrp{\rho g \cos \alpha \phi - \gamma \kappa} = \grad_2 P
    \label{eq:grad2_p}
\end{align}
where $P = \rho g \cos \alpha \phi - \gamma \kappa$. Substituting this into \cref{eq:uv_bar} gives 
\begin{align}
    \bar{\vect{v}} = -\frac{h^2}{3\mu} \lrp{\grad P - \rho g \sin \alpha \vect{i}}
    \label{eq:uv_bar_final}
\end{align}
where $\grad_2$ is replaced by $\grad$ since $P$ is only a function of $x$ and $y$. 

The conservation of mass, when depth-averaged, gives 
\begin{align}
    \pderiv{h}{t} + \grad \cdot \lrp{h\bar{\vect{v}}} = 0
\end{align}
which results in 
\begin{align}
    \nonumber \pderiv{h}{t} &= \frac{1}{3\mu} \grad \cdot \lrb{h^3 \lrp{\grad P - \rho g \sin \alpha \vect{i}}}\\
    &= \frac{1}{3\mu} \grad \cdot \lrb{h^3 \lrp{\rho g \cos\alpha \grad \phi - \gamma\grad\kappa - \rho g \sin \alpha \vect{i}}}. 
\end{align}
Approximating the curvature $\kappa \approx \grad^2 \phi$ then gives 
\begin{align}
    \nonumber \pderiv{h}{t} &= \frac{1}{3\mu} \grad \cdot \lrb{h^3 \lrp{\rho g \cos\alpha \grad \phi - \gamma\grad\grad^2\phi - \rho g \sin \alpha \vect{i}}}\\
    %&= \frac{1}{3\mu} \grad \cdot \lrb{\rho g \cos\alpha h^3\grad (h+s) - \gamma h^3\grad\grad^2(h+s) - h^3\rho g \sin \alpha \vect{i}}.
    &= \frac{1}{3\mu} \lrb{ \grad \cdot \lrb{\rho g \cos \alpha h^3 \grad \phi} - \grad \cdot \lrb{\gamma h^3 \grad\grad^2\phi} - \rho h \sin\alpha \pderiv{h^3}{x} }.
    \label{eq:thin_film_dim}
\end{align}

\subsection{Dimensionless Form}
Scale the height of the free surface by $\bar{\phi} = \phi / h_c$ and rescale the in-plane coordinates and time by 
$\lrp{\bar{x}, \bar{y}, \bar{t}} = \lrp{x/x_c, y/x_c, t/t_c}$. Using these scales, \cref{eq:thin_film_dim} after removing the bars turns into 
\begin{align}
    \nonumber \frac{h_c}{t_c} \pderiv{h}{t} &= \frac{1}{3\mu} \lrb{ \frac{1}{x_c} \grad \cdot \lrb{ \rho g \cos\alpha \frac{h_c^4}{x_c} h^3 \grad \phi } - \frac{1}{x_c} \grad \cdot \lrb{ \gamma \frac{h_c^4}{x_c^3}h^3 \grad\grad^2\phi } - \rho g \sin\alpha \frac{h_c^3}{x_c} \pderiv{h^3}{x} }\\
    \nonumber &= \frac{1}{3\mu} \lrb{ \frac{h_c^4}{x_c^2}\rho g \cos\alpha \grad \cdot \lrb{ h^3 \grad \phi } - \frac{\gamma h_c^4}{x_c^4} \grad \cdot \lrb{h^3 \grad\grad^2\phi } - \rho g \sin\alpha \frac{h_c^3}{x_c} \pderiv{h^3}{x} }\\
    \pderiv{h}{t} &= \frac{\gamma h_c^3 t_c}{3\mu x_c^4}\lrb{ \frac{x_c^2 \rho g}{\gamma} \cos\alpha \grad \cdot \lrb{ h^3 \grad \phi } - \grad \cdot \lrb{h^3 \grad\grad^2\phi } - \frac{x_c^3\rho g}{\gamma h_c} \sin\alpha \pderiv{h^3}{x} }.
    \label{eq:thin_film_dimless_step1}
\end{align}
Choose $x_c$ and $t_c$ such that
\begin{align}
    x_c = \lrp{\frac{\gamma h_c}{\rho g}}^{1/3}, \quad t_c = \frac{3\mu x_c}{h_c^2 \rho g}.
    \label{eq:x_and_t_scales}
\end{align} 
Substituting the scales in \cref{eq:x_and_t_scales} into \cref{eq:thin_film_dimless_step1} yields
\begin{align}
    \pderiv{h}{t} = \lrp{\frac{h_c^2 \rho g}{\gamma}}^{1/3} \cos\alpha \grad \cdot \lrb{ h^3 \grad \phi } - \grad \cdot \lrb{h^3 \grad\grad^2\phi } - \sin\alpha \pderiv{h^3}{x}. 
    \label{eq:thin_film_dimless_step2}
\end{align} 
The velocity scale is chosen naturally as $U = x_c/t_c = h_c^2\rho g/3\mu$,
and we can additionally define the capillary number $Ca = \mu U / \gamma = h_c^2\rho g/3\gamma$. 
Hence, \cref{eq:thin_film_dimless_step2} becomes 
\begin{align}
    \nonumber \pderiv{h}{t} &= D\cos\alpha \grad \cdot \lrb{h^3\grad\phi} - \grad \cdot \lrb{h^3\grad\grad^2\phi} - \sin\alpha \pderiv{h^3}{x}\\
    &= D\cos\alpha \grad \cdot \lrb{h^3\grad\lrp{h+s}} - \grad \cdot \lrb{h^3\grad\grad^2\lrp{h+s}} - \sin\alpha \pderiv{h^3}{x}
    \label{eq:thin_film_dimless}
\end{align}
where $D = \lrp{3Ca}^{1/3}$. 