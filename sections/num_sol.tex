\subsection{Spatial Discretization}
If we define the domain of interest as $\lrb{0, L_x}$, we can discretize the 
domain into points $x_j = j\Delta x$ for $j = 0, \ldots, N_x$ where 
$\Delta x = L_x / N_x$ and $N_x$ is the number of grid points excluding the origin. 
If we further define $\func{h_j}{t} = \func{h}{x_j, t}$, we can discretize 
\cref{eq:two_dim_final} into a system of ordinary differential equations of the form 
\begin{equation}
    \deriv{h_j}{t} = f_j = \mathrm{Bo}\cos \beta f_j^{(1)} - f_j^{(2)} -  \frac{\mathrm{Bo}}{\varepsilon} \sin \beta f_j^{(3)} +  
    \frac{k_i \lrp{1 + \alpha_1^2} \mathrm{We_{ac}}}{\varepsilon} f_j^{(4)}, \quad j = 0, \ldots, N_x
    \label{eq:ode_sys}
\end{equation}
where $f_j^{(k)}$ is the discretization of the $k$-th term in
the right-hand side of \cref{eq:two_dim_final}. Because the governing equation 
contains high order derivatives, the discretization used for certain components 
needs special attention in order to not lead to a large computational stencil.

\subsubsection{Fourth-Order Term}
Following the method outlined in \cite{kondic2003instabilities}, discretizing the fourth order term (i.e.\! $f_j^{(2)}$) can be done by a combination
of forward and backward differences. 

\subsubsection{Lower Order Terms}
