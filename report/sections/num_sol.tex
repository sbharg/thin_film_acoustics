\subsection{Spatial Discretization}
If we define the domain of interest as $\lrb{0, L_x}$, we can discretize the 
domain into points $x_j = j\Delta x$ for $j = 0, \ldots, N_x$ where 
$\Delta x = L_x / N_x$ and $N_x$ is the number of grid points excluding the origin. 
If we further define $\func{h_j}{t} = \func{h}{x_j, t}$, we can discretize 
\cref{eq:two_dim_final} into a system of ordinary differential equations of the form 
\begin{equation}
    \deriv{h_j}{t} = f_j = \mathrm{Bo}\cos \beta f_j^{(1)} - f_j^{(2)} -  \frac{\mathrm{Bo}}{\varepsilon} \sin \beta f_j^{(3)} +  
    \frac{k_i \lrp{1 + \alpha_1^2} \mathrm{We_{ac}}}{\varepsilon} f_j^{(4)}, \quad j = 1, \ldots, N_x-1
    \label{eq:ode_sys}
\end{equation}
where $f_j^{(k)}$ is the discretization of the $k$-th term in
the right-hand side of \cref{eq:two_dim_final} and the ODEs for $j = 0, N_x$ are prescribed by boundary conditions. 
Because the governing equation 
contains high order derivatives, the discretization used for certain components 
needs special attention in order to not lead to a large computational stencil.

\subsubsection{Fourth-Order Term}
Following the method outlined in \cite{kondic2003instabilities}, discretizing the fourth order term (i.e.\! $f_j^{(2)}$) can be done by a combination
of forward and backward differences. If we define 
\begin{align*}
    h_{x, j} = \frac{h_{j+1} - h_j}{\Delta x} \quad \quad \quad h_{\overline{x}, j} = \frac{h_{j} - h_{j-1}}{\Delta x}
\end{align*}
as the forward and backward differences, respectively, then a possible discretization is 
\begin{equation}
    f_j^{\lrp{2}} = \lrp{\func{a}{h_{j-1}, h_j} \phi_{\overline{x}x\overline{x}, j}}_{x, j}
    \label{eq:f2_disc}
\end{equation}
where $\func{a}{j_1, j_2} = \frac{1}{2} \lrp{{j_1}^3 + {j_2}^3}$. This discretization leads to a 
second order approximation that has a five point stencil, which is better than the seven point stencil
that would result from using a central differencing scheme. 

\subsubsection{Lower Order Terms}
Because the other terms are lower order, we can apply simpler differencing methods as long as 
they are also second order and don't increase the size of the stencil. For example, we could apply 
a combination of forward and backward differencing to get 
\begin{align}
    f_j^{\lrp{1}} = \lrp{\func{a}{h_{j-1}, h_j}\phi_{x, j}}_{\bar{x}, j}. 
\end{align}
If we define central differencing as 
\begin{equation*}
    h_{x^*, j} = \frac{h_{j+1} - h_{j-1}}{2\Delta x}
\end{equation*}
then the other two terms could be discretized as 
\begin{gather}
    f_j^{\lrp{3}} = \lrp{h^3_j}_{x^*, j}\\
    f_j^{\lrp{4}} = \lrp{h^3_j e^{2k_i \lrp{x_j + \alpha_1 \varepsilon \phi_j}} \lrp{1 + \alpha_1 \varepsilon \phi_{x^*, j}}}_{x^*, j}. 
\end{gather}
A complete expansion of these discretized terms is provided in \cref{sec:space_discrete_expand} of the appendix. 

\subsection{Time stepping}
Because the governing PDE describing the system is stiff (i.e. some numerical methods will be unstable unless the step size taken for time evolution is extremely small), 
we find that using explicit schemes do not work for time stepping. Thus, we choose to use an implicit scheme called Rodas4 implemented 
in the DifferntialEquations.jl library in Julia. 