\section{Governing Equation}
\subsection{Incompressible Navier-Stokes}
\begin{frame}
    We start with the Incompressible Navier-Stokes Equation
    \begin{multline}
        \rho \lrp{\pderiv{\vecu}{t} + \lrp{\vecu \cdot \grad}\vecu} = -\grad p
        + \mu \grad^2 \vecu + \rho g \sin \beta \vect{i} - \rho g \cos \beta \vect{k} \\
        - \rho J e^{2k_i \lrp{x + \alpha_1 z}} \vect{i} - \rho J \alpha_1 e^{2k_i \lrp{x + \alpha_1 z}} \vect{k}
        \label{eq:ns-eq}
    \end{multline}
    where 
    \begin{itemize} 
        \item $\vect{u} = \lrp{u,v,w} =$ Fluid velocity, $p =$ Fluid pressure, $\rho =$ Fluid density, $\mu =$ Fluid viscosity
        \item $k_i =$  Attenuation coefficient, $\alpha_1 =$ Geometric constant, and $J = \lrp{1 + \alpha_1^2}A^2\omega^2 k_i$
        is a constant we define to consolidate terms
    \end{itemize}

    %The first two vector terms in the $x$ and $z$ direction represent the in plane and out of plane 
    %components of gravity, while the last two vector terms represent the in plane and out of plane 
    %components of the SAW forcing. 
\end{frame}
\subsection{Lubrication Approximation}
\begin{frame}
    The Lubrication Approximation assumes we are dealing with thin films
    and allows us to ignore the inertial terms of the Navier-Stokes equation (LHS) as well as the in plane
    derivatives and normal component of $\vecu$. Hence, \cref{eq:ns-eq} reduces to 
    \begin{equation}
        \begin{aligned}
            \grad_2 p &= \mu \frac{\partial^2 \vect{v}}{\partial z^2} + \rho g \sin\beta \vect{i} - \rho J\cexpz\vect{i}\\
            \pderiv{p}{z} &= -\rho g \cos\beta - \rho J\alpha_1 \cexpz
        \end{aligned}
        \label{eq:lub_approx}
    \end{equation}
    where $\grad_2 = \lrp{\partial_x, \partial_y}$ and $\vect{v} = \lrp{u, v}$. 
\end{frame}
\subsection{Boundary Conditions}
\begin{frame}
    We use the following boundary conditions 
    \begin{itemize}
        \item Laplace-Young: At the interface $z = \func{\phi}{x, y, t}$ the pressure is given by $\func{p}{\phi} = -\gamma \kappa + p_0$
        where $\kappa$ is the curvature of the boundary, $\gamma$ is the surface tension, and $p_0$ is the atmospheric pressure
        \item Vanishing shear stresses: $\pderiv{\vect{v}}{z} = \vect{0}$ along $z = \func{\phi}{x, y, t}$
        \item No slip: $\vect{v} = \vect{0}$ along the surface $z = \func{s}{x, y}$
    \end{itemize}

    Using these conditions and averaging over the height gives 
    \begin{equation*}
        \bar{\vect{v}} = -\frac{h^2}{3\mu} \lrp{\rho g \cos\beta \grad \phi - \gamma\grad\kappa - \rho g \sin\beta\vect{i} + \frac{\rho J}{2k_i}\grad \cexpp}.
    \end{equation*}
\end{frame}
\subsection{Dimensional Equation}
\begin{frame}
    The conservation of mass, when depth-averaged, gives 
    \begin{equation*}\pderiv{h}{t} + \grad \cdot \lrp{h\bar{\vect{v}}} = 0.\end{equation*}
    Approximating $\kappa \approx \grad^2 \phi$, this gives a final dimensional equation 
    \begin{multline}
        \pderiv{h}{t} = \frac{1}{3\mu} \lrb{ \grad \cdot \lrb{\rho g \cos \beta h^3 \grad \phi} - \grad \cdot \lrb{\gamma h^3 \grad\grad^2\phi}} \\ + \frac{1}{3\mu} \lrb{- \rho g \sin\beta \pderiv{h^3}{x} + \grad \cdot \lrb{\frac{\rho J}{2k_i}h^3\grad \cexpp}}.
        \label{eq:thin_film_dim}
    \end{multline}
\end{frame}
\subsection{Dimensionless Equation - 2 Spatial Dimensions}
\begin{frame}
    Scaling the coordinates and time by 
    \begin{equation*}
        \bar{x} = \frac{x}{x_c}, \quad \bar{y} = \frac{y}{x_c}, \quad \bar{z} = \frac{z}{h_c}, \quad \bar{t} = \frac{t}{t_c}
    \end{equation*}
    to get dimensionless quantities and using the following characteristics
    \begin{equation*}
        t_c = \frac{3\mu x_c^3}{\gamma h_c^3}, \quad \varepsilon = \frac{h_c}{x_c}, \quad \mathrm{Bo} = \frac{x_c^2\rho g}{\gamma}, \quad \mathrm{We_{ac}} = \frac{\rho \omega^2 A^2x_c}{\gamma}
    \end{equation*}
    gives the dimensionless equation 
    \begin{multline}
        \pderiv{h}{t} = \mathrm{Bo} \cos\beta \grad \cdot \lrb{ h^3 \grad \phi } - 
        \grad \cdot \lrb{h^3 \grad\grad^2\phi } - 
        \frac{\mathrm{Bo}}{\varepsilon} \sin\beta \pderiv{h^3}{x} \\ + 
        \frac{\lrp{1 + \alpha_1^2} \mathrm{We_{ac}}}{2\varepsilon} \grad \cdot \lrb{h^3 \grad e^{2k_i \lrp{x + \alpha_1 \varepsilon \phi}}} 
        \label{eq:nondim_final_twod}
    \end{multline} 
    after removing any overlines.   
\end{frame}
\subsection{Dimensionless Equation - 1 Spatial Dimension}
\begin{frame}
    We make the further simplification that the free surface of the film does not change
    in the transverse direction (i.e.\! $h$ and $s$ are both $y$-independent). 
    This simplifies \cref{eq:nondim_final_twod} to 
    \begin{multline}
        \pderiv{h}{t} = \mathrm{Bo}\cos\beta \lrb{h^3 \phi_x}_x - \lrb{h^3 \phi_{xxx}}_{x} - \frac{\mathrm{Bo}}{\varepsilon} \sin\beta \lrb{h^3}_{x} \\ + 
        \frac{k_i \lrp{1 + \alpha_1^2}\mathrm{We_{ac}}}{\varepsilon} \lrb{h^3 e^{2k_i \lrp{x + \alpha_1 \varepsilon \phi}} \lrp{1 + \alpha_1 \varepsilon \phi_x}}_x.
        \label{eq:nondim_final_oned}
    \end{multline}

    To enforce that the SAW forcing occurs starting from the film front on the left, we redefine 
    $k_i$ (in dimensionless form) as 
    \begin{equation*}
        \func{k_i}{h} = x_c \lrp{\lrp{k_i^{\text{liquid}} - k_i^{\text{air}}}\lrp{1 - e^{-x_c(h-b)/\lambda}} + k_i^{\text{air}}}.
        \label{eq:k_i}
    \end{equation*}
\end{frame}