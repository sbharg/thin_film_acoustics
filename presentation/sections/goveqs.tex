\section{Governing Equation}
\subsection{Incompressible Navier-Stokes}
\begin{frame}
    We start with the Incompressible Navier-Stokes Equation
    \begin{multline}
        \rho \lrp{\pderiv{\vecu}{t} + \lrp{\vecu \cdot \grad}\vecu} = -\grad p
        + \mu \grad^2 \vecu + \rho g \sin \beta \vect{i} - \rho g \cos \beta \vect{k} \\
        - \rho J e^{2k_i \lrp{x + \alpha_1 z}} \vect{i} - \rho J \alpha_1 e^{2k_i \lrp{x + \alpha_1 z}} \vect{k}
        \label{eq:ns-eq}
    \end{multline}
    where $\vect{u} =$ Fluid velocity, $p =$ Fluid pressure, $\rho =$ Fluid density, $\mu =$ Fluid viscosity, 
    $k_i =$  Attenuation coefficient, $\alpha_1 =$ Geometric constant, and $J = \lrp{1 + \alpha_1^2}A^2\omega^2 k_i$
    is a constant we define to consolidate terms. 

    The first two vector terms in the $x$ and $z$ direction represent the in plane and out of plane 
    components of gravity, while the last two vector terms represent the in plane and out of plane 
    components of the SAW forcing. 
\end{frame}
\subsection{Lubrication Approximation}
\begin{frame}
    The Lubrication Approximation assumes we are dealing with thin films
    and allows us to ignore the inertial terms of the Navier-Stokes equation (LHS) as well as the in plane
    derivatives and normal component of $\vecu$. Hence, \cref{eq:ns-eq} reduces to 
    \begin{equation}
        \begin{aligned}
            \grad_2 p &= \mu \frac{\partial^2 \vect{v}}{\partial z^2} + \rho g \sin\beta \vect{i} - \rho J\cexpz\vect{i}\\
            \pderiv{p}{z} &= -\rho g \cos\beta - \rho J\alpha_1 \cexpz
        \end{aligned}
        \label{eq:lub_approx}
    \end{equation}
    where $\grad_2 = \lrp{\partial_x, \partial_y}$ and $\vect{v} = \lrp{u, v}$. 
\end{frame}
\subsection{Boundary Conditions}
\begin{frame}

\end{frame}