\section{Numerical Simulation}
\subsection{Discretization}
\begin{frame}
    If we define the domain of interest as $\lrb{0, L_x}$, we can discretize the 
    domain into $N_x + 1$ points such that
    \begin{equation*}
        x_j = j\Delta x \quad \quad j = 0, \ldots, N_x \quad \quad \Delta x = L_x / N_x. 
    \end{equation*}
    We further define $\func{h_j}{t} = \func{h}{x_j, t}$ which allows us to discretize 
    (temp eq holder) into a system of $N_x + 1$ ODEs of the form 
    \begin{multline}
        \deriv{h_j}{t} = \mathrm{Bo}\cos \beta f_j^{(1)} - f_j^{(2)} -  \frac{\mathrm{Bo}}{\varepsilon} \sin \beta f_j^{(3)} 
        \\ + \frac{k_i \lrp{1 + \alpha_1^2} \mathrm{We_{ac}}}{\varepsilon} f_j^{(4)}
        \label{eq:ode_sys}
    \end{multline}
    where $f_j^{(k)}$ is the discretization of the $k$-th term. 
\end{frame}
\begin{frame}
    Discretizing $f_j^{(2)}$ (i.e. the fourth order term) requires the most care 
    in order to not lead to a large computational stencil. Using a combination of forward
    and backward differences
    \begin{equation*}
        h_{x, j} = \frac{h_{j+1} - h_j}{\Delta x} \text{ (Forward)} \quad \quad \quad h_{\overline{x}, j} = \frac{h_{j} - h_{j-1}}{\Delta x} \text{ (Backward)}
    \end{equation*}
    then a possible discretization is 
    \begin{equation*}
        \lrb{h^3\phi_{xxx}}_x \approx f_j^{\lrp{2}} = \lrp{\frac{1}{2}\lrp{h_{i-1}^3 + h_i^3} \phi_{\overline{x}x\overline{x}, j}}_{x}
    \end{equation*}
    which is a second order approximation that gives a five point stencil. 
\end{frame}
\begin{frame} 
    Because the other terms are lower order, we can apply simpler differencing methods
    and not worry about increasing the size of our stencil. 
    \begin{gather*}  
        \lrb{h^3\phi_x}_x \approx f_j^{\lrp{1}} = \frac{1}{2}\lrp{\lrp{h_{i-1}^3 + h_i^3} \phi_{x, j}}_{\overline{x}}\\
        \lrb{h^3}_x \approx f_j^{\lrp{3}} = \lrp{h^3_j}_{x^*}\\
    \end{gather*}
    \vspace{-4em}
    \begin{multline*}
        \lrb{h^3 e^{2k_i \lrp{x + \alpha_1 \varepsilon \phi}} \lrp{1 + \alpha_1 \varepsilon \phi_x}}_x 
        \approx f_j^{\lrp{4}} = \\ \lrp{h^3_j e^{2k_i \lrp{x_j + \alpha_1 \varepsilon \phi_j}} \lrp{1 + \alpha_1 \varepsilon \phi_{x^*, j}}}_{x^*}
    \end{multline*}
\end{frame}